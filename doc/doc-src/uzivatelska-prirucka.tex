\documentclass[a4paper]{article}
\usepackage[utf8]{inputenc}
\usepackage[czech]{babel}
\usepackage{ifpdf}
\usepackage{hyperref}
\title{allCHEMIE -- Uživatelská příručka}
\author{Ondrej Profant}
\date{\today}
\ifpdf
\hypersetup{
	pdfauthor={Ondrej Profant},
	pdftitle={allCHEMIE - Uživatelská příručka},
}
\fi
\begin{document}
\maketitle
\newpage
\tableofcontents
\newpage
\section{Úvod}
Program allChemie vznikl jako zápočtový roku 2010 a je psán v~jazyce Prolog. Slouží k~základní práci se vzorci převážně anorganické chemie. Přesněji řečeno má tyto základní možnosti užití:
\begin{enumerate}
\item Informace o~jednotlivých prvcích
\item Koncovky českého názvosloví, řecké přípony etc.
\item Převod chem. vzorce na slovní vyjádření (CO$_2$ -> Oxid uhličitý)
\item Ze slovního vyjádření určí vzorec (Oxid uhličitý -> CO$_2$)
\item Doplňování možností (CX$_N$ -> CO$_2$)

\end{enumerate}
\section{Instalace} 
\subsection{Linux}
Je třeba nainstalovat \href{swi-prolog}{http://www.swi-prolog.org/}, např. v~Debianu/Ubuntu:
\begin{verbatim}
sudo apt-get install swi-prolog
\end{verbatim}
Ten pak spustíme příkazem:

\begin{verbatim}
prolog
\end{verbatim}

to nás, ale úplně nezajímá. My chceme spustit konkrétně allCHEMII:
\begin{verbatim}
prolog -q -f <cesta_k_souboru/main.pro>
\end{verbatim}
(očekává se, že všechny soubory programu budou v~dané složce)
\subsection{Windows}
Stačí stáhnout a nainstalovat \href{swi-prolog}{http://www.swi-prolog.org/}, poté stáhnout zdrojové kódy programu, rozbalit a kliknout na soubor "main". Program allChemie pak bude asociovat s~Prologem, čili můžete začít pracovat s~programem podle dalšího návodu. Správnou instalaci programu allChemie poznáte tak, že se po kliknutí na soubor "main" se zobrazí hlavička s~nápisem CHEMIE a s~možnostmi využití programu.
\section{Použití}
\subsection{Predikáty v~Prologu}
Varování: Tato kapitola bude asi velmi těžko pochopitelná technicky méně zdatným uživatelům, ale při čtení dalšího textu bude velmi užitečná.

Program není „klasickým program", ale spíše „skriptem" pro jazyk Prolog. Je dobré uvědomit si několik základních faktů:
\begin{itemize}
\item Spuštěním prologu se dostaneme do interaktivního prostředí, kde bude: „?- "~a kurzor. V~tomto prostředí zadáváme predikáty.
\item Predikát je „vlastně funkce": má název a za ním v~závorce parametry
\begin{itemize}
\item Pokud ho zadáváme, tak za ním musíme napsat tečku.
\end{itemize}
\item Enterem predikát potvrdíme a pak se Prolog pokusí vyhovět takovémuto predikátu. Pokud nám výsledek nevyhovuje můžeme zmáčknout středník a dostaneme další možnosti.
\item Parametry jsou:
\begin{itemize}
\item Velká písmena, což jsou proměnné, které chceme doplnit 
\item Malá písmena, což jsou pevně daná fakta (data)
\item Podtržítko, což je taková proměnná, co nás nezajímá (anonymní)
\end{itemize}
\end{itemize}
\subsection{Informace o~jednotlivých prvcích}
Stačí zadat: \texttt{vstup(Prvek)}, kde za Prvek dosadíte značku prvku malými písmeny, např:
\begin{verbatim}
vstup(h).
\end{verbatim}
Pro vyhledávání dle více parametrů slouží predikát prvek, který má 6 parametrů. Ovládá se jako každý predikát v~Prologu. Malé písmeno znamená pevně daný fakt (například parametr vyhledávání), velké písmeno znamená proměnnou (tedy to, co chceme najít) a podtržítko znamená údaj, který nás nezajímá. Syntaxe predikátu prvek je následující:
\begin{verbatim}
prvek( <prot. cislo> , <znacka> , <nazev> , {kov,pol,nek} , <skupina> , <sezn. oxid. cisel>). 
\end{verbatim}
, kde \{\} značí množinu - jsou zde právě tři možnosti. Čili použití následující:
\begin{verbatim}
prvek( 1 , h , vodik , _ , _ , X ).
\end{verbatim}
V~odpovědi se nám dostane \texttt{X}, které bude seznamem oxidačních čísel, kterých nabývá vodík, tedy:
\begin{verbatim}
X = [1].
\end{verbatim}

\subsection{Koncovky českého názvosloví}
Koncovky se používají obdobně jako predikát prvek z~předchozí kapitoly. Používá se predikát koncovka, jehož syntaxe je následující:
\begin{verbatim}
koncovka( <cislo> , <pripon.> , <pripon. kys> , <pripon. sůl> ).
\end{verbatim}
\subsection{Převod chem. vzorce na slovní vyjádření}
Zde je důležitý predikát \texttt{vstup()}, který je silně přetížený - tj. má proměnný počet parametrů (2-10). Stačí mu zadat kompletní vzorec (pro správnou funkci je třeba vyplnit všechna čísla vč. 1). Např:
\begin{verbatim}
vstup(h,2,o,1).                 voda
vstup(c,1,o,2).                 oxid uhličitý
vstup(cl,2,o,7).                oxid chloristý
vstup(ba,1,(oh),2).             hydroxid barnatý
vstup(h,1,f,1).                 flurovodík
vstup(h,2,s,1,o,4).             kyselina sírová
vstup(h,3,p,1,o,4).             kyselina trihydrogen fosforečná
vstup(na,1,cl,1).               chlorid sodný
vstup(k,1,n,1,o,3).             dusicnan draselný
vstup(nh4,2,c,1,o,3).           uhlicitan amonný
vstup(na,1,h,2,p,1,o,4,1).      dihydrogenfosforecnan sodný
vstup(ca,1,s,1,o,4,1).          síran vápenatý
vstup(ca,1,s,1,o,4,1,*,5,h2o).  pentahydrát síranu vápenatého
\end{verbatim}
Některé sloučeniny mají nadefinované zjednodušené vzory, ale samozřejmě to také zvyšuje pravděpodobnost výskytu chyby (res. špatné interpretace), např:
\begin{verbatim}
vstup(h,2,o).          voda
vstup(c,o,2).          oxid uhličitý
vstup(h,f).            flurovodík
vstup(ba,oh,2).        hydroxid barnatý
vstup(h,2,s,o,4).      kyselina sírová
\end{verbatim}
\subsubsection{Malá poznámka k~solím}
Amoniak (NH3), res jeho kationt (NH4) je považován za atom a vepisuje se vcelku.

Poslední číslo u~solí je obvykle dobrovolné a udává počet celé aniontové skupiny, např v: \texttt{Ca$_1$(SO$_4$)$_1$} je to jedna.
\subsection{Organika}
\subsection{Doplnění neúplného vzorce o~možné prvky a četnosti}
K~této funkci je využíván samotný Prolog, stačí nahradit prvek proměnnou a Prolog již najde možné prvky.
\begin{verbatim}
vstup(X,1,o,2).
X = c ;         oxid uhličitý
X = n ;         oxid dusičitý
X = si ;        oxid křemičitý
X = s ;         oxid siřičitý
...
\end{verbatim}
Vynechat lze i více prvků, či četnost nějakého atomu.
\subsection{Skládání sloučenin}
Predikát \texttt{sloz()} jednoduše najde možné sloučeniny. Četnosti si doplňuje sám. 
\begin{verbatim}
sloz(cl,na).
chlorid sodný

sloz(na,cl).
chlorid sodný

sloz(h,o).
voda ;
peroxid vodiku.
\end{verbatim}
Je připraven na četnost 2,3,4.
\section{Známé problémy}
\section{Otázky a odpovědi (FAQ)}
\section{Zdroje}
\end{document}
