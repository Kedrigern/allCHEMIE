\documentclass[a4paper]{article}
\usepackage[utf8]{inputenc}
\usepackage[czech]{babel}
\usepackage{ifpdf}
\usepackage{hyperref}
\title{allCHEMIE -- Programátorská dokumentace}
\author{Ondrej Profant}
\date{\today}
\ifpdf
\hypersetup{
	pdfauthor={Ondrej Profant},
	pdftitle={allCHEMIE - Programátorská dokumentace},
}
\fi
\begin{document}
\maketitle
\newpage
\tableofcontents
\newpage
\section{Úvod}
Celý program je velmi úzce spjat s~programovacím jazykem Prolog, který spadá do paradigmatu logického programování.
To znamená, že kód se skládá z~predikátů nad databází již zadaných predikátů a snaží se doplnit všechny proměnné. 
Nicméně věřím, že není nutné rozumnět Prologu pro to, aby jste si AllCHEMIi vylepšili či pochopili, co dělá.
Zvlátště pak odborné části z~chemie jsou vysloveně jednoduché, složitější může být občas aritmetika dopočítávání sloučenin. 

Pokud Prolog znáte, tak celý kód je velmi jasně napsaný a nejsou v~něm žádné zaludnosti ani složitá rekurze.

V~další kapitole je podrobně popsána tvorba predikátu pro výpočet oxidu.
\section{Tvorba vlastního predikátu} 
\subsection{Jednoduché podmínky}
První a velmi důležitým rozhodnutím je, jak vlastně bude náš predikát deklarován. 
Celý program dodržuje konvenci, že všechny výpočty jsou deklarovány jako predikát \texttt{vstup()}, ale
jak určit jaké bude mít argumenty již není tak jednoduché. Pojďme si vytvořit pravidla pro oxidy.

Oxid bude mít jistě 4 argumenty, dva budou prvky a dva počty výskytů. Navíc víme, že druhý prvek bude vždy
kyslík. Oxid tedy bude zadefinován takto:

\begin{verbatim}
vstup( Prvek, N, o, M ). % zatim bez jakyhkoliv pravidel
\end{verbatim}

Důležité je dodržovat zásady jazyka Prolog -- proměnná je vždy velkým písmenem. Za znakem procenta je komentář, tam
si můžete napsat libovolnou poznámku. Dalším jednoduchým omezením (omezujeme množinu všech oxidů na ten, který nám byl zadán)
jsou číselné rozsahy. Víme, že kationt se v~oxidu vyskytuje jednou nebo dvakrát a oxid může: 1,2,3,4,5,7. 
To zapíšeme takto:

\begin{verbatim}
member(N,[1,2]),
member(M,[1,2,3,4,5,7]),
\end{verbatim}

Další věcí, kterou můžeme snadno ověřit jsou informace o~prvku. Na to máme predikát \texttt{prvek()/6}. 
U~oxidu ověříme, že prvek není v~A8 skupině a jakých může nabývat oxid. čísel. Také si rovnou najdeme jeho 
plné jméno.

\begin{verbatim}
prvek( _, Prvek, Jmeno, _, Sk, Cisla ),
not(Sk == a8), % dulezite je male a v a8, jinak by se jednalo o promennou
\end{verbatim}

V~jiných případech ještě můžeme ověřit zda je prvek kov, polokov či nekov, zde to nemá smysl. A~nyni stačí
již jen dopočítat zbylé oxidační číslo prvku.

\subsection{Rovnice (vyčíslení)}
Nejdříve budeme pokračovat v~předchozím příkladu vypočtu oxdidů a pak se zmíním o~obecnějších věcech.

U~oxidu není složitého dopočítat poslední číslo, neb víme:

\begin{verbatim}
Pstrana is(abs(-2*M)), %absolutni hod. prave strany
X is(Pstrana / N), % (Lstrana := X*N) == Pstrana
nabyvaOxCisla(Jmeno,X),
\end{verbatim}

Pokud dojdeme až sem, tak jsme korektně vypočíslili oxid, nyní ho stačí vypsat:

\begin{verbatim}
!, %tzv. řez, zabraní Prologu hledat další alternativy
priponSt(X,Koncovka), 
write('oxid '),write(Jmeno),write('-'),write(Koncovka).
\end{verbatim}

Aritmetika není zrovna silnou a oblíbenou stránkou Prologu. Celou dobu jsme vlastně pracovali s~textovými
řetězci (včetně: \texttt{memeber(N,[1,2,3])}) a až speciální predikát \texttt{is()}\footnote{
Dokumentace SWI-Prologu: http://www.swi-prolog.org/pldoc/man?predicate=is\%2f2}  převedl řetězce na čísla
a začal s~nimi počítat. Jak vidíte, tak se chová normálně operátory jsou: \texttt{$+$}, \texttt{$-$}, \texttt{*},
\texttt{/}, \texttt{//} (celočíselné), \texttt{=:=} (rovná se), \texttt{=\\=} (nerovná se).

\subsection{Kompletace}
Nyní doděláme nezbytnosti okolo, ještě budeme potřebovat:
\begin{verbatim}
:-consult('tabulka.pl'). %pripoji tabulku prvků.
\end{verbatim}
Celý kód bude tedy vypadat takto:
\begin{verbatim}
:-consult('tabulka.pl'). %pripoji tabulku prvků.

vstup( Prvek, N, o, M ) :-
   member(N,[1,2]),
   member(M,[1,2,3,4,5,7]),
   prvek( _, Prvek, Jmeno, _, Sk, Cisla ),
   not(Sk == a8), % dulezite je male a v a8, jinak by se jednalo o promennou
   Pstrana is(abs(-2*M)), %absolutni hod. prave strany
   X is(Pstrana / N), % (Lstrana := X*N) == Pstrana
   nabyvaOxCisla(Jmeno,X),
   !, %tzv. řez, zabraní Prologu hledat další alternativy
   priponSt(X,Koncovka), 
   write('oxid '),write(Jmeno),write('-'),write(Koncovka).
\end{verbatim}

Co s~kódem udělat? Otevřeme si libovolný textový editor (ve Windows třeba Notepad)
a vložíme ho. Soubor uložíme např. jako \texttt{mujOxid.pl} a ve stejné složce máme
soubor \texttt{tabulka.pl}. Pak již soubor \texttt{mujOxid.pl} stačí spustit v~Linuxu
z~terminálu:
\begin{verbatim}
prolog -f mujOxid.pl
\end{verbatim}
ve Windows poklepáním na daný soubor. Samozřejmě je nutné mít nainstalovaný SWI-Prolog.

\section{Soubory a logické členění programu}
\end{document}
